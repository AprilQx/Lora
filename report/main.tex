\documentclass{article}
\usepackage[utf8]{inputenc}
\usepackage{CJKutf8}
\usepackage{graphicx}
\usepackage{subfig}
\usepackage{float}

\usepackage{tikz}
\usepackage{xcolor}

% Custom colors
\definecolor{process0}{RGB}{144,202,249}  % Light blue
\definecolor{process1}{RGB}{165,214,167}  % Light green
\definecolor{process2}{RGB}{255,204,128}  % Light orange
\definecolor{process3}{RGB}{206,147,216}  % Light purple
\definecolor{halocell}{RGB}{239,154,154}  % Light red

\usepackage{dirtree}

% Layout
\usepackage[left=1.5in, right=1.5in, top=1in, bottom=1in]{geometry}

% Images
\graphicspath{{./images/}}
\usepackage{wrapfig}
\usepackage{tabularx}
\usepackage{multirow}
\usepackage{totalcount}




% Math
\usepackage{amsmath,amssymb,amsthm,enumitem,bm}

\DeclareMathOperator{\sech}{sech}
\DeclareMathOperator{\csch}{csch}
\DeclareMathOperator{\arcsec}{arcsec}
\DeclareMathOperator{\arccot}{arcCot}
\DeclareMathOperator{\arccsc}{arcCsc}
\DeclareMathOperator{\arccosh}{arcCosh}
\DeclareMathOperator{\arcsinh}{arcsinh}
\DeclareMathOperator{\arctanh}{arctanh}
\DeclareMathOperator{\arcsech}{arcsech}
\DeclareMathOperator{\arccsch}{arcCsch}
\DeclareMathOperator{\arccoth}{arcCoth}

\usepackage{listings}
\usepackage{parskip}
\setlength{\parindent}{0pt}

\begin{document}
% Title
\begin{center}
    \huge\textbf{M2: Lora Finetuning}
\end{center}



\begin{center}
    \Large Xueqing Xu (xx823)
    
    Department of Physics, University of Cambridge
    
    April 7, 2025
\end{center}

\section*{Introduction}
\begin{itemize}
    \item Brief overview of the coursework objectives
    \item Introduction to LoRA (Low-Rank Adaptation) and its application to LLMs
    \item Introduction to time series forecasting using language models
    \item Mention of the predator-prey system (Lotka-Volterra) as the target application
\end{itemize}
This coursework explores the application of Low-Rank Adaptation (LoRA) to fine-tune the Qwen2.5-Instruct Large Language Model (LLM) for forecasting predator-prey population dynamics. Building on the observation that LLMs can function as time series forecasters without explicit training\cite{gruver2023large}, we investigate how targeted fine-tuning can enhance their forecasting capabilities while maintaining efficiency.

LoRA represents a parameter-efficient fine-tuning technique that dramatically reduces the number of trainable parameters by injecting small, trainable low-rank matrices into existing model weights without modifying the original parameters. This approach is particularly valuable when working with large models under computational constraints.

The target application is forecasting the Lotka-Volterra predator-prey system, a classic ecological model that describes the dynamic interaction between two species. This system exhibits oscillatory behavior that presents a challenging forecasting task requiring understanding of non-linear dynamics and the ability to model complex interdependencies.
\section*{Methodology}
\subsection*{Qwen2.5-Instruct model architecture}
The Qwen2.5-0.5B-Instruct model implements a decoder-only transformer architecture with 494 million parameters (Table \ref{tab:qwen25-overview}). With a hidden dimension of 896 across 24 transformer layers (Table \ref{tab:qwen25-architecture}), the model balances depth and computational efficiency.

A key architectural feature is Grouped Query Attention (GQA), which employs 14 query heads but only 2 key-value heads—a 7:1 ratio that substantially reduces memory usage during inference while maintaining attention capabilities. For normalization, Qwen2.5 uses RMSNorm with $\epsilon = 10^{-6}$, which offers better training stability than traditional LayerNorm.
% Qwen2.5-0.5B-Instruct Model Overview Table
\begin{table}[H]
\centering
\begin{tabular}{c c}
\hline
\textbf{Property} & \textbf{Value} \\
\hline
Model Name & Qwen2.5-0.5B-Instruct \\
Model Type & qwen2 \\
Total Parameters & 494,032,768 ($\sim$0.5B) \\
Architecture & Decoder-only Transformer \\
Precision & bfloat16 \\
\hline
\end{tabular}
\caption{Qwen2.5-0.5B-Instruct Model Overview}
\label{tab:qwen25-overview}
\end{table}
% Add this where you want the references to appear

\bibliographystyle{plain}
\bibliography{references}
\end{document}